%%=============================================================================
%% Voorwoord
%%=============================================================================

\chapter*{\IfLanguageName{dutch}{Woord vooraf}{Preface}}
\label{ch:voorwoord}

%% TODO:
%% Het voorwoord is het enige deel van de bachelorproef waar je vanuit je
%% eigen standpunt (``ik-vorm'') mag schrijven. Je kan hier bv. motiveren
%% waarom jij het onderwerp wil bespreken.
%% Vergeet ook niet te bedanken wie je geholpen/gesteund/... heeft

Deze bachelorproef is het sluitstuk van de opleiding Bachelor in de Toegepaste Informatica aan de Hogeschool Gent en is gericht op de specialisatie Systeem- en Netwerkbeheer. De inspiratie voor Chaos Engineering te onderzoeken vond ik in het verhaal van Netflix die doelbewust hun applicatie en systemen ging aanvallen om te leren hoe zij vervolgens verbeteringen konden aanbrengen om de weerbaarheid te verhogen. Tijdens mijn opleiding leerde ik alreeds hoe moderne applicaties via de microservices softwarearchitectuur worden opgesplitst in meerdere containers en hoe deze applicaties via een CI/CD pipeline kunnen getest worden om alreeds vroegtijdig fouten op te sporen. Hoe deze applicaties kunnen groeien via containerorkestratie zoals Kubernetes en de uitdagingen die dit met zich meebrengt was dan ook een logisch vervolg in mijn leertraject. Door zowel Kubernetes als Chaos Engineering te onderzoeken leerde ik meer bij over de interne werking van Kubernetes en hoe deze acties onderneemt wanneer applicaties of onderdelen van de cluster getroffen worden. 

Gedurende dit onderzoek kon ik meermaals rekenen op hulp en advies van mijn promotor Gertjan Bosteels en co-promotor Bert Van Vreckem. Mijn oprechte dank gaat uit naar hen voor alle ondersteuning en tijd die zij in mij geïnvesteerd hebben om dit onderzoek tot een goed eind te brengen.

Verder wil ik ook mijn oneindige dank betuigen aan mijn vrouw, familie en vrienden die me ondersteund hebben en het mogelijk maakten de opleiding Bachelor in de Toegepaste Informatica als afstandsstudent succesvol te combineren met een vaste job en een gezinsleven met twee jonge kinderen. Zonder hun onvoorwaardelijke steun was ik nooit tot dit punt in de opleiding kunnen raken.      