%%=============================================================================
%% Cloud Kubernetes clusters
%%=============================================================================

\chapter{\IfLanguageName{dutch}{Cloud Kubernetes clusters}{Cloud Kubernetes clusters}}
\label{ch:cloudclusters}

\section{Google Cloud Platform}

In het voorgaande hoofdstuk \ref{ch:lokaleclusters} lag de focus op het in stand brengen van een Kubernetes cluster in een lokale omgeving. Initieel werd de voorkeur gegeven aan een lokale omgeving omdat deze dichter aanleunt bij de manier hoe men in schoolverband een virtuele omgeving tot stand brengt, vaak met behulp van automatisatie. Ondanks de automatisatie bleek dit nog steeds een tijdrovende taak te zijn. 

In dit hoofdstuk zal besproken worden hoe men een Kubernetes cluster kan opzetten via Google Cloud. Hiervoor zal Google Kubernetes Engine (GKE) gebruikt worden, een Kubernetes beheer- en orkestratieplatform van Google. GKE voorziet een beheerde omgeving voor het uitrollen, beheren en schalen van gecontaineriseerde applicaties.

Na de installatieprocedure zal ook kort ingegaan worden op de extra functionaliteit die het opzetten van een GKE cluster met zich meebrengt. 

\subsection{Opzetten van een GKE cluster via Google Cloud}
\label{sec:cloudclustersetup}

Om een cluster op te zetten via Google Cloud doet men beroep op Google Kubernetes Engine (GKE). Nieuwe klanten kunnen GKE eerst uittesten door een account te creëeren waarbij een gratis krediet van driehonderd dollar gegeven wordt, geldig voor een periode van negentig dagen. Men zal bij de creatie van een account een kredietkaart of andere betalingswijze moeten opgeven, om een Cloud Billing account op te zetten en de identiteit te bevestigen. Indien het gratis krediet eerder op is of de negentig dagen zijn verstreken, dan zal men uitdrukkelijk moeten upgraden naar een betaalde Cloud Billing account om verder gebruik te kunnen maken van de diensten die GKE aanbiedt. \autocite{GoogleCloud2022} 

Via volgende stappenplan kan men een account op GKE creëeren en een cluster opzetten: 
\begin{enumerate}
    \item Ga naar GKE via volgende URL: \url{https://cloud.google.com/kubernetes-engine}
    \item Vul uw gegevens in + activeer uw account.
    \item Kies voor de optie {\bf Enable} bij de Kubernetes Engine API en wacht vervolgens één a twee minuten tot het startscherm van het Google Cloud Platform tevoorschijn komt. Bovenaan deze pagina zal u het huidige saldo en de resterende dagen steeds kunnen opvolgen.
    \item Centraal op deze pagina ziet men {\bf Kubernetes Engine - Kubernetes Clusters}. Kies hier eerst voor de optie {\bf Create}, en vervolgens de optie {\bf GKE Standard} om verder te gaan. 
    \item De cluster die aan de hand van dit stappenplan wordt opgezet zal bestaan uit drie nodes. Voldoende resources toewijzen aan de cluster is cruciaal, aangezien men later zowel pods zal installeren toebehorend aan een chaos engineering tool als twee demo-applicaties. {\bf Cluster Autoscaling} zal ook toegepast worden op de cluster en zal o.b.v. de vraag naar resources extra nodes toevoegen of nodes afbouwen. \autocite{GKE2022} \newline Nota: Hierdoor is het mogelijk dat in het verloop van de experimenten er meer of minder dan drie nodes actief zullen zijn. 
    \newline Volgende configuraties kan men toepassen om de cluster op te zetten:
    \begin{enumerate}
        \item {\bf Een Machine Type definiëren:} Ga in het linkermenu eerst naar {\bf Node pools} en vervolgens onder {\bf default-pool} naar het tabblad {\bf Nodes}. Een goede richtlijn is per node 2 vCPU en 4GB RAM toe te wijzen. Dit kan men bekomen door in de dropdownlist {\bf Machine Type} te kiezen voor {\bf e2-medium (2vCPU, 4 GB memory)}.
        \item {\bf Enable cluster autoscaler activeren}: Vink deze optie aan om er voor 
        te zorgen dat GKE de nodes in de cluster automatisch zal schalen. 
        \item Klik onderaan op Create om de cluster te creëeren. Dit proces zal ongeveer één a twee minuten tijd nodig hebben om af te ronden. Eens de cluster operationeel is zal men naast de naam bij Status een groene vink zien staan.
        \newline Men bevindt zich nu in het Kubernetes Engine menu in het Google Cloud Platform. Dit menu kan u terugvinden onder het dropdownmenu in de linkerbovenhoek (naast Google Cloud Platform).
    \end{enumerate}    
    \item Klik op de naam van de cluster om het overzicht te openen. Daar ziet men algemene configuraties van deze cluster.
    \item Om een terminalsessie in de cluster te openen kiest men rechtsbovenaan in de taakbalk voor de optie {\bf Connect}. Een nieuw scherm {\bf Connect to the cluster} zal hierdoor geopend worden.
    \item Kies voor de optie {\bf Run in Cloud Shell}. Dit zal automatisch het bovenstaande commando meenemen naar een nieuwe terminalsessie. Daar kan men vervolgens het commando uitvoeren.
    \item Een nieuw scherm {\bf Authorize Cloud Shell} wordt hierdoor geopend, waar men via de optie {\bf Authorize} de toegang krijgt tot de cluster.   
\end{enumerate} 

Op een identieke manier als bij het opzetten van de lokale Kubernetes clusters, kan men hier ook snel een eerste demo-applicatie lanceren om de verdeling van de pods over de nodes te controleren. Gebruik hiervoor commando {\bf kubectl create deploy nginx --image=nginx --replicas=3}.

Uit de output van het commando {\bf kubectl get pods -o wide} kan men zien dat de drie applicatie-pods verdeeld zijn over de drie beschikbare nodes. Dit was eveneens het geval bij de laatste lokale cluster setup via Kubespray, maar met het fundamentele verschil dat in deze cloudomgeving de control-plane node beheerd wordt door Google zelf, waardoor bij de creatie van de cluster enkel worker nodes in de node pool aanwezig zijn. Men kan via commando {\bf kubectl cluster-info} informatie omtrent de control plane node opvragen. 

\subsection{Extra functionaliteit in GKE}
\label{subsec:extrafuncgke}
De Deployments en pods van de demo-applicaties, alsook enkele gemonitorde metrics zoals CPU- en geheugengebruik, kan men opvolgen door in het linkermenu naar Workloads te gaan. Wanneer men een Deployment selecteert in dit menu kan men bovenaan in de menubalk enkele handige opties zien die het toelaten om configuraties rechtstreeks vanuit de browser uit te voeren i.p.v. via de terminal. De opties hier zijn o.a.:
\begin{itemize}
    \item {\bf Edit:} De YAML-configuratie van de Deployment aanpassen.
    \item {\bf Delete:} De Deployment verwijderen.
    \item {\bf Actions:} 
        \subitem {\bf Autoscale:} Een HorizontalPodAutoscaler configureren. (zie Hoofdstuk~\ref{subsec: HPA}) 
        \subitem {\bf Expose:} Een Service creëeren.
        \subitem {\bf Rolling Update:} De versie van de container image vernieuwen.
        \subitem {\bf Scale:} Instellen hoeveel pods de ReplicaSet van de Deployment actief moet houden. Men kan hier eveneens een limiet instellen op de resources die pods in de Deployment mogen gebruiken. 
    \item {\bf Kubectl:} Het benodigde commando om de YAML-configuratie van de Deployment op te vragen rechtstreeks inladen in de Cloud Shell terminal.
\end{itemize}

Nota: Ondanks de aanwezigheid van deze functionaliteit zal gedurende dit onderzoek voornamelijk vanuit de Cloud Shell terminal gewerkt worden om soortgelijke configuraties te bekomen. De reden hiervoor is om bekend te raken met de commando's waarmee Kubernetes objecten geconfigureerd worden.

\subsubsection{GKE monitoring}

In het Workloads menu kan men alreeds enkele metrics opvolgen, maar Google Kubernetes Engine (GKE) heeft ook een volwaardig monitoring platform. Wanneer men in GKE in het Clusters menu de cluster selecteert gaat men naar tabblad {\bf Details}. Scroll in dit overzicht tot aan de sectie {\bf Features} waar men men in de lijst {\bf Cloud Monitoring} kan terugvinden. Men kan naar het monitoring platform gaan door te klikken op de link {\bf View GKE Dashboard}.

In de pagina die geopend wordt kan men heel wat functionaliteit opmerken. In het linkermenu ziet men {\bf Metrics Explorer}. Via deze kan men verschillende metrics opvragen van de nodes in de cluster. 
Dit zal later van pas komen in het chaos engineering experiment waarbij node resources zoals geheugen/CPU belast worden. 

Nota: Een alternatief pad naar dit platform kan ook genomen worden door in één van de opgevolgde metrics in het Workloads menu de optie rechtsbovenaan 'More chart options' te selecteren en vervolgens voor 'View in Metrics Explorer' te kiezen. 

Voor het opvolgen van andere experimenten zal de terminal-gebaseerde UI {\bf k9s} gebruikt worden, waarvan de setup alreeds besproken is in Hoofdstuk~\ref{subsec:k9s}. 
  
\subsection{Conclusie GKE cluster}

Het opzetten van een GKE cluster via Google Cloud is heel envoudig en verloopt veel sneller ten opzichte van een lokale Kubernetes setup op te zetten via de eerder besproken opties in dit onderzoek. Ondanks een cluster opzetten in GKE niet gratis is kan men deze omgeving wel eerst uittesten via het gratis krediet van 300 dollar die men krijgt bij de creatie van een account. Dit krediet is geldig gedurende negentig dagen waardoor dit platform in aanmerking komt om te gebruiken voor educatieve doeleinden. Het enige nadeel om een account te bekomen is dat een betaalwijze moet meegegeven worden, ondanks men wel uitdrukkelijk verduidelijkt dat géén bedrag zal gefactureerd worden zonder men eerst aangeeft over te gaan naar een betalend abonnement. 

GKE biedt veel functionaliteit, waaronder de mogelijkheid tot het automatisch schalen van nodes via de autoscaling-optie, de mogelijkheid om configuraties vanuit de browser uit te voeren, de aanwezigheid van een monitoring platform om metrics op te volgen ... Dit is echter nog maar een kleine greep uit het enorme aanbod waardoor men wel de nodige tijd moet investeren om wegwijs te raken in GKE.  