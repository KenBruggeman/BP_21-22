%%=============================================================================
%% Conclusie
%%=============================================================================

\chapter{Conclusie}
\label{ch:conclusie}

% TODO: Trek een duidelijke conclusie, in de vorm van een antwoord op de
% onderzoeksvra(a)g(en). Wat was jouw bijdrage aan het onderzoeksdomein en
% hoe biedt dit meerwaarde aan het vakgebied/doelgroep? 
% Reflecteer kritisch over het resultaat. In Engelse teksten wordt deze sectie
% ``Discussion'' genoemd. Had je deze uitkomst verwacht? Zijn er zaken die nog
% niet duidelijk zijn?
% Heeft het onderzoek geleid tot nieuwe vragen die uitnodigen tot verder 
%onderzoek?
Gedurende dit onderzoek werd een antwoord gezocht op de vragen uit Hoofdstuk~\ref{sec:onderzoeksvraag}. Volgende antwoorden kan men aanschouwen als de conclusie van dit onderzoek:  

\begin{enumerate}
\item {\bf Biedt het een meerwaarde om een lokale Kubernetes cluster op te zetten m.b.v. automatisatietools ten opzichte van een Kubernetes cluster via Google Cloud op te zetten?}

Verschillende lokale Kubernetes clusters werden opgezet tijdens dit onderzoek. Een single-node Minikube cluster kan snel en eenvoudig opgezet worden maar biedt geen meerwaarde om het inzicht in de werking van Kubernetes te verruimen. Eveneens kan men geen experimenten toepassen die gericht zijn op een node, aangezien dit de enige node in de omgeving zou treffen. Een multi-node cluster opzetten via Kubeadm is een tijdrovende manuele taak, maar geeft wel het nodige inzicht welke stappen nodig zijn om een cluster operationeel te krijgen. Deze manuele taken konden grotendeels opgelost worden via de geautomatiseerde setup m.b.v. de Ansible playbook in Kubespray, maar deze had bijna een half uur nodig om de cluster op te zetten wat opnieuw als tijdrovend kan aanschouwd worden. Een monitoring platform opzetten in een lokale omgeving is om deze reden niet verder onderzocht.
\newline De snelheid en het gemak waarmee een cluster opgezet wordt in Google Cloud is ten opzichte van een lokale setup een enorm verschil. Een cluster opzetten in Google Cloud duurt slechts enkele minuten. Men kan extra functionaliteit toevoegen tijdens de setup zoals autoscaling, waardoor nodes automatisch kunnen schalen. Eveneens is er de aanwezigheid van een monitoring dashboard in Google Cloud, waardoor sommige experimenten zoals het belasten van het geheugen beter opgevolgd kunnen worden. Google Cloud biedt 300 dollar aan gratis krediet aan gedurende negentig dagen. Hierdoor is het mogelijk deze setup te gebruiken in een curriculum systeem-en netwerkbeheer.     
\newline 
\item {\bf Welke chaos engineering tool die in dit onderzoek aan bod komt is het meest geschikt om experimenten uit te voeren op een applicatie in een Kubernetes cluster?}
\newline Het antwoord op deze vraag is minder eenvoudig want elke onderzochte tool heeft namelijk zijn voor- en nadelen. Chaos Toolkit en Litmus opzetten ging vrij vlot, maar de GUI van Chaos Mesh bracht tijdens de setup de nodige problemen met zich mee. Chaos Toolkit kan enkel gebruikt worden via de terminal en heeft een beperkt aanbod qua experimenten ten opzichte van de andere onderzochte tools. De tool moet uitgebreid worden via verschillende extensies/plugins om meer functionaliteit te creëeren. Het voordeel bij Chaos Toolkit is dat de structuur, de uitvoer en het resultaat van de experimenten in de terminal vrij duidelijk zijn. Deze tool is eveneens goed gedocumenteerd. Puur bekeken op experimenten uitvoeren via de terminal zou de voorkeur naar deze tool gaan.
\newline Chaos Mesh en Litmus hebben een breed gamma aan experimenten die zowel uitgevoerd kunnen worden via de terminal als via de GUI. Ondanks experimenten uitvoeren in de terminal weinig meerwaarde biedt ligt de moeilijkheidsgraad bij Litmus om dit te realiseren veel hoger dan bij Chaos Mesh. De experimenten bij Chaos Mesh zijn beter gedocumenteerd dan bij Litmus en de leercurve bij Chaos Mesh is ook minder steil. Litmus heeft bovendien meer resources nodig om operationeel te kunnen zijn tegenover Chaos Mesh. Op vlak van functionaliteit bij de uitvoer van experimenten is Litmus wel beter voorzien dan Chaos Mesh en Chaos Toolkit. Als documentatie en leercurve de doorslag kunnen geven in de vergelijking geniet Chaos Mesh de voorkeur. Het combineren van deze tools is echter perfect mogelijk. In dat opzicht kan men eventueel het beste van beiden combineren en zowel Chaos Toolkit als Chaos Mesh gebruiken om experimenten op te zetten.   
\newline 
\item {\bf Welke chaos engineering experimenten zijn relevant om een beter inzicht te creëeren in de werking van Kubernetes?}  
\newline Dit onderzoek is vertrokken vanuit het standpunt om een betere basiskennis in Kubernetes te creëeren via deze experimenten. Het had dan ook weinig zin om nodeloos complexe applicaties op te zetten die een gevorderde kennis van Kubernetes zouden vereisen. Door de applicatie(s) simpel te houden konden alreeds enkele relevante zaken omtrent de werking van Kubernetes aangetoond worden zoals o.a.: 
\begin{itemize}
    \item het vernietigen van pods in een Deployment triggert de ReplicaSet om nieuwe pods te creëeren.
    \item het vernietigen van pods in de Podtato-Head applicatie bracht aan het licht dat het voordeliger is om meerdere pods via een Deployment te creëeren, zodat een applicatie bereikbaar blijft wanneer een pod getroffen wordt.
    \item het simuleren van netwerkproblemen toonde het belang aan van de communicatie tussen Services die verantwoordelijk zijn voor de bereikbaarheid van pods verspreid over verschillende nodes.      
    \item het geheugen belasten van pods kan opgevangen worden via een HorizontalPodAutoscaler die extra pods creëert om deze belasting op te vangen.
    \item het simuleren van een nodefaling toonde aan dat Kubernetes alle pods op de getroffen node kon verhuizen. 
\end{itemize} 
\newline 
\item {\bf Heeft dit onderzoek geleid tot nieuwe vragen die uitnodigen tot verder onderzoek?}
\newline Door de tijdsdruk gedurende dit onderzoek zijn mogelijks interessantere lokale setups van een Kubernetes cluster die minder tijdrovend zijn niet aan bod gekomen. Het opzetten van een monitoring platform in een lokale omgeving is evenmin aan bod gekomen en vereist verder onderzoek. Eveneens zijn er nog interessante open-source chaos engineering tools niet onderzocht, die mogelijks beter geschikt zijn voor educatieve doeleinden. \newline Zowel de zoektocht naar een vlottere manier om een lokale omgeving op te zetten, alsook een geschikte chaos engineering tool te vinden om te gebruiken in deze omgeving, nodigt zich uit tot verder onderzoek.   
\end{enumerate}