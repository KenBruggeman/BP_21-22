%%=============================================================================
%% Samenvatting
%%=============================================================================

% TODO: De "abstract" of samenvatting is een kernachtige (~ 1 blz. voor een
% thesis) synthese van het document.
%
% Deze aspecten moeten zeker aan bod komen:
% - Context: waarom is dit werk belangrijk?
% - Nood: waarom moest dit onderzocht worden?
% - Taak: wat heb je precies gedaan?
% - Object: wat staat in dit document geschreven?
% - Resultaat: wat was het resultaat?
% - Conclusie: wat is/zijn de belangrijkste conclusie(s)?
% - Perspectief: blijven er nog vragen open die in de toekomst nog kunnen
%    onderzocht worden? Wat is een mogelijk vervolg voor jouw onderzoek?
%
% LET OP! Een samenvatting is GEEN voorwoord!

%%---------- Nederlandse samenvatting -----------------------------------------
%
% TODO: Als je je bachelorproef in het Engels schrijft, moet je eerst een
% Nederlandse samenvatting invoegen. Haal daarvoor onderstaande code uit
% commentaar.
% Wie zijn bachelorproef in het Nederlands schrijft, kan dit negeren, de inhoud
% wordt niet in het document ingevoegd.

\IfLanguageName{english}{%
\selectlanguage{dutch}
\chapter*{Samenvatting}

\selectlanguage{english}
}{}

%%---------- Samenvatting -----------------------------------------------------
% De samenvatting in de hoofdtaal van het document

\chapter*{\IfLanguageName{dutch}{Samenvatting}{Abstract}}

Kubernetes is een bekende vorm van containerorkestratie en heeft alreeds een enorme impact gehad op het IT-landschap. Om het inzicht in de werking ervan te verruimen kan gebruik gemaakt worden van Chaos Engineering, die toelaat proactief te experimenteren op een omgeving zodoende een systeem beter te leren kennen. Dit onderzoek richt zich op de educatieve mogelijkheden om chaos engineering tools en experimenten toe te passen in een Kubernetes cluster. Eerst is onderzocht als het voordelig is een cluster lokaal op te zetten via automatisatietools. Drie proof of concepts zijn uitgewerkt nl. een single-node cluster via Minikube en een multi-node cluster via Kubeadm en Kubespray.    
 
Door de tijdrovende procedure om lokale clusters tot stand te brengen drong de vergelijking zich op met een cloudomgeving. Als gevolg werd een cluster via Google Kubernetes Engine (GKE) opgezet. De conclusie is dat een cluster in de cloud opzetten veel sneller verloopt en eveneens extra functionaliteiten bevat die handig zijn in het verder verloop van het onderzoek naar een geschikte chaos engineering tool.  

Vervolgens werden in deze cloudomgeving de chaos engineering tools Chaos Toolkit, Chaos Mesh en Litmus vergeleken. Met de nodige voorzichtigheid werd Chaos Mesh gekozen aangezien deze een ruim aanbod aan experimenten bevat, goed gedocumenteerd is, weinig resources vereist in een cluster, en experimenten relatief makkelijk op te zetten en uit te voeren zijn zowel in de terminal als via de browser.

De experimenten in dit onderzoek werden uitgevoerd op twee demo-applicaties en konden aantonen hoe Kubernetes actie onderneemt wanneer het geconfonteerd wordt met realistische problemen die een applicatie in productie kunnen treffen. Eveneens kwamen door het uitvoeren van deze experimenten enkele oplossingen tot stand om de beschikbaarheid van de applicatie(s) te vergroten.    
